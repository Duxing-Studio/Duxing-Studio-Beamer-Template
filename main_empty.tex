% \documentclass[handout]{beamer}
\documentclass[presentation]{beamer}
\usepackage{ctex}
\usecolortheme{duxing}
\usefonttheme{duxing}
\useinnertheme{duxing}
 
\usepackage[utf8]{inputenc}
\usepackage[UKenglish]{babel}
\usepackage{booktabs}
\usepackage{caption}
\usepackage{subcaption}
\usepackage{graphicx}
\usepackage{amsmath}
\usepackage{amsfonts}
\usepackage{amssymb}

\institute{\includegraphics[height=0.7cm]{sysu_logo.png}\includegraphics[height=0.7cm]{duxing_logo.png}}

\title{\hspace{-13pt}\includegraphics[width=0.5\linewidth]{duxing_logo.png}}

\subtitle{发展数学爱好,提升学术素养}

\author{副总监}

\date{}


\begin{document}
 
\frame{\titlepage}

\section{前言}

\begin{frame}[standout]
    在谈到如果要给后进的学弟学妹一个学习的方法的话,刘路回答到:
    
    “我只有一个想法,就是看淡分数,重在兴趣。”
\end{frame}

\begin{frame}[standout]
    发展数学爱好,提升学术素养
\end{frame}

\section{部门简介}

\begin{frame}
    \frametitle{数研组}\centering
    将能通过数学专题讲座或是课后讨论班的形式接触到许多高于课堂的数学知识。
\end{frame}

\begin{frame}
	\frametitle{研发组}\centering
	学习编程与人工智能相关知识,比如有深度学习研讨班深入学习深度学习知识,侧重于数学在计算机领域的应用。
\end{frame}

\begin{frame}
	\frametitle{统计组}\centering
	学习统计学知识、金融相关知识和部分常用编程软件的简单入门等。
\end{frame}

\begin{frame}
	\frametitle{运营组}\centering
	协助工作室管理日常的运作,举办属于自己的特色活动,负责在秋季学期末的时候举办数模新手赛。
\end{frame}

\begin{frame}[standout]
    Questions
\end{frame}

\end{document}

